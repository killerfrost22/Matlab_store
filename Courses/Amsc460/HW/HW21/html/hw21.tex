
% This LaTeX was auto-generated from MATLAB code.
% To make changes, update the MATLAB code and republish this document.

\documentclass{article}
\usepackage{graphicx}
\usepackage{color}

\sloppy
\definecolor{lightgray}{gray}{0.5}
\setlength{\parindent}{0pt}

\begin{document}

    
    
\subsection*{Contents}

\begin{itemize}
\setlength{\itemsep}{-1ex}
   \item AMSC 460 - HW21
   \item Problem 1
   \item Problem 2
\end{itemize}


\subsection*{AMSC 460 - HW21}

\begin{verbatim}
clear all; format compact; close all; syms f(x) x y z
\end{verbatim}


\subsection*{Problem 1}

\begin{par}
If using the composite trapezoid rule, how many subintervals are needed to approximate the integral to within an absolute error of $10^{-8}$?
\end{par} \vspace{1em}
\begin{verbatim}
img = imread('amsc460p1.jpg'); imshow(img)
\end{verbatim}

\includegraphics [width=4in]{hw21_01.eps}
\begin{verbatim}
syms n
vpasolve(9*exp(3)/(12*n^2)== 10^(-8))
\end{verbatim}

        \color{lightgray} \begin{verbatim}ans =
-38812.565867758281377226918861775
 38812.565867758281377226918861775
\end{verbatim} \color{black}
    

\subsection*{Problem 2}

\begin{par}
Implement the composite trapezoid rule in MATLAB. For $n = 10^p$, with $p=1,2,3,4$, keep track of the error $E_T(n) = |I-T(f,n)|$, where $T(f,n)$ denotes the trapezoid rule with $n$ subintervals. Plot $E_T$ for the given values of $n$ on the same plot (use logarithmic axes for better scaling). Does your plot agree with the analysis performed in part (a)? \ensuremath{\backslash}\ensuremath{\backslash}
\end{par} \vspace{1em}



\end{document}

