
% This LaTeX was auto-generated from MATLAB code.
% To make changes, update the MATLAB code and republish this document.

\documentclass{article}
\usepackage{graphicx}
\usepackage{color}

\sloppy
\definecolor{lightgray}{gray}{0.5}
\setlength{\parindent}{0pt}

\begin{document}

    
    
\subsection*{Contents}

\begin{itemize}
\setlength{\itemsep}{-1ex}
   \item AMSC 460 - HW16
   \item Problem 1
\end{itemize}


\subsection*{AMSC 460 - HW16}

\begin{verbatim}
clear all; format compact; close all; syms f(x) x y z
\end{verbatim}


\subsection*{Problem 1}

\begin{par}
(a) Using MATLAB's backslash command, find the solution to the normal equations $A^T Ax = A^T b$.
\end{par} \vspace{1em}
\begin{verbatim}
A = [1+10^(-8) -1;-1 1;1 -1]
b = [2;1;-1]
\end{verbatim}

        \color{lightgray} \begin{verbatim}A =
    1.0000   -1.0000
   -1.0000    1.0000
    1.0000   -1.0000
b =
     2
     1
    -1
\end{verbatim} \color{black}
    \begin{verbatim}
AT = A.';
B = AT*A;
C = AT*b;
x1 = B\C
\end{verbatim}

        \color{lightgray} \begin{verbatim}Warning: Matrix is close to singular or badly scaled. Results may be
inaccurate. RCOND =  2.896879e-17. 
x1 =
   1.0e+07 *
    5.7533
    5.7533
\end{verbatim} \color{black}
    \begin{par}
(b)Using MATLAB's \ensuremath{\backslash}texttt\{qr\} command and backslash, find the solution to the triangular system $\widehat{R} x = (Q^T b)_{1:n}$. Compare this to the solution to the normal equations. How far apart are the answers?  Compute the distance in a norm of your choice.
\end{par} \vspace{1em}
\begin{verbatim}
[Q,R] = qr(A)
\end{verbatim}

        \color{lightgray} \begin{verbatim}Q =
   -0.5774   -0.8165   -0.0000
    0.5774   -0.4082    0.7071
   -0.5774    0.4082    0.7071
R =
   -1.7321    1.7321
         0   -0.0000
         0         0
\end{verbatim} \color{black}
    \begin{verbatim}
Qt = Q.'
D = Qt*b
x2 = R\D
distance = abs(x1-x2)
norm(x1-x2)
\end{verbatim}

        \color{lightgray} \begin{verbatim}Qt =
   -0.5774    0.5774   -0.5774
   -0.8165   -0.4082    0.4082
   -0.0000    0.7071    0.7071
D =
   -0.0000
   -2.4495
   -0.0000
x2 =
   1.0e+08 *
    3.0000
    3.0000
distance =
   1.0e+08 *
    2.4247
    2.4247
ans =
   3.4290e+08
\end{verbatim} \color{black}
    \begin{par}
(c)Using MATLAB's \ensuremath{\backslash}texttt\{cond\} command, what are the condition numbers of $A$, $A^T A$, and $\widehat{R}$? Which condition number should we worry about in double precision floating point arithmetic? Which computed answer is more accurate?
\end{par} \vspace{1em}
\begin{verbatim}
cond(A)
\end{verbatim}

        \color{lightgray} \begin{verbatim}ans =
   4.2426e+08
\end{verbatim} \color{black}
    \begin{par}
Condition number of $A$, \$A\^{}T
\end{par} \vspace{1em}
\begin{verbatim}
cond(B)
cond(R)
% Since all of the items are much larger than 1 so both moethods are bad
% Therefore the QR decomposition is more accurate
\end{verbatim}

        \color{lightgray} \begin{verbatim}ans =
   1.7998e+16
ans =
   4.2426e+08
\end{verbatim} \color{black}
    


\end{document}

