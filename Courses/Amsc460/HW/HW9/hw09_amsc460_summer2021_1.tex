\documentclass[12pt]{article}
\usepackage{hyperref,color,palatino,amsmath,amssymb,graphicx}
\usepackage[normalem]{ulem} %For striking out text
\usepackage[margin=1in]{geometry}

\hypersetup{
    colorlinks=true,
    linkcolor=blue,
    filecolor=magenta,      
    urlcolor=cyan,
}
\urlstyle{same}

\pagenumbering{gobble}

\begin{document}

\begin{center}
Computational Methods\qquad Summer 2021
\\

\textbf{\large HOMEWORK 9}\\
\end{center}
\noindent \textbf{Due Date}: Tuesday, June 15\\

\begin{enumerate}
\item Rearrange the following equations to form a strictly diagonally dominant system. Apply two steps of Jacobi and Gauss-Seidel methods starting with the zero vector. (Do the computations by hand.)
	\begin{equation*}\begin{aligned}
	& u+3v=-1,\\ 
	& 5u+4v=6.
	\end{aligned}\end{equation*}
\item Componentwise the Jacobi method for solving $A\textbf{x}=\textbf{b}$ reads
	\begin{equation}x_{i}^{(k+1)}=\frac{1}{a_{ii}}\left[b_i -\sum_{j=1}^{i-1}a_{ij}x_j^{(k)}-\sum_{j=i+1}^n a_{ij}x_j^{(k)}\right],\end{equation}
where $x_i^{(k)}$ denotes the $i^{th}$ component of the $k^{th}$ Jacobi iterate, for $i=1,...,n$.

In many practial problems the matrix $A$ is \emph{sparse}, with many of its entries being zero. Iterative methods can work well for such problems since the zeros don't need to be stored, and operation counts are reduced. A usual example are banded matrices such as the following tridiagonal matrix with bandwidth 3,
	\begin{equation}A=\begin{pmatrix}
	d_1&u_1&0&0&0\\
	l_2&d_2&u_2&0&0\\
	0&l_3&d_3&u_3&0\\
	0&0&l_4&d_4&u_4\\
	0&0&0&l_5&d_5\end{pmatrix}.\end{equation}
Only the $d_i, l_i, u_i$ entries need to be stored. The notation is extended in an obvious way to matrices of size $n\times n$.

For this problem, rewrite the Jacobi method (1) for a tridiagonal matrix of order $n$, using the notation introduced in (2). In doing so, suppose that only the tridiagonal (nonzero) terms are stored and can be accessed. As an  example, neither $a_{1,3}$ nor $a_{3,1}$ should be accessed, and your revision should reflect that.


\end{enumerate}
\end{document}