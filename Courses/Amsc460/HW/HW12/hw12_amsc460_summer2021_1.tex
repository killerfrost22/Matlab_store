\documentclass[12pt]{article}
\usepackage{hyperref,color,palatino,amsmath,amssymb,graphicx,enumerate}
\usepackage[normalem]{ulem} %For striking out text
\usepackage[margin=1in]{geometry}

\hypersetup{
    colorlinks=true,
    linkcolor=blue,
    filecolor=magenta,      
    urlcolor=cyan,
}
\urlstyle{same}

\pagenumbering{gobble}

\begin{document}

\begin{center}
Computational Methods\qquad Summer 2021
\\

\textbf{\large HOMEWORK 12}\\
\end{center}
\noindent \textbf{Due Date}: Tuesday, June 22\\

\begin{enumerate}
\item Use Newton's divided differences to find the Newton table, and a polynomial for interpolating the points $(-1,0),(2,1),(3,1),(5,2)$.
\item Let $P(x)$ be the degree 10 polynomial taking the value zero at  $x=1,2,...,10$ and where $P(12)=66$. Find $P(0)$. (Hint: Choose the interpolant basis wisely.)
\item (Optional, not graded) Count the number of operations ($+-*/$) needed to evaluate a polynomial through $n$ data points in (i) Lagrange form, and (ii) Newton's divided differences. Which is more efficient in terms of evaluation complexity? [Suppose in (ii) that the Newton polynomial is written in \emph{nested form}. For example the nested form of $1+2x+3x^2+4x^3$ is $1+x(2+x(3+4x))$.]
\end{enumerate}
\end{document}