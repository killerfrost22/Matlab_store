
% This LaTeX was auto-generated from MATLAB code.
% To make changes, update the MATLAB code and republish this document.

\documentclass{article}
\usepackage{graphicx}
\usepackage{color}

\sloppy
\definecolor{lightgray}{gray}{0.5}
\setlength{\parindent}{0pt}

\begin{document}

    
    
\subsection*{Contents}

\begin{itemize}
\setlength{\itemsep}{-1ex}
   \item AMSC 460 - HW11
   \item Problem 1
   \item Problem 2 (
\end{itemize}


\subsection*{AMSC 460 - HW11}

\begin{verbatim}
clear all; format compact; close all; syms f(x) x y z
\end{verbatim}


\subsection*{Problem 1}

\begin{par}
Use Newton's divided differences to find the Newton table, and a polynomial for interpolating the points $(-1,0),(2,1),(3,1),(5,2)$.
\end{par} \vspace{1em}
\begin{par}
P(x) = 1/3*(x+1)-1/12(x+1)(x-2) + 1/24(x+1)(x-2)(x-3)
\end{par} \vspace{1em}
\begin{par}
= (1/3)x+1/3 -(1/12)x\^{}2+(1/12)x+1/6 +(1/24)x\^{}3-(1/6)x\^{}2+(1/24)x+1/4
\end{par} \vspace{1em}
\begin{par}
= (1/24)x\^{}3-(1/4)x\^{}2+(11/24)x+3/4
\end{par} \vspace{1em}


\subsection*{Problem 2 (}

\begin{par}
Let $P(x)$ be the degree 10 polynomial taking the value zero at  $x=1,2,...,10$ and where $P(12)=66$. Find $P(0)$. (Hint: Choose the interpolant basis wisely.)
\end{par} \vspace{1em}
\begin{par}
P(x) = a(x-1)(x-2)(x-3)...(x-10) and P(12) = 66 P(12) = a(12-1)(12-2)(12-3)...(12-10) =66
\end{par} \vspace{1em}
\begin{par}
a *11! = 66 so we have a = 66/11!
\end{par} \vspace{1em}
\begin{par}
Therefore P(x) = 66/11!(x-1)(x-2)(x-3)...(x-10) So we can say P(0)= (66/11!)*10!=66/11=6
\end{par} \vspace{1em}



\end{document}

