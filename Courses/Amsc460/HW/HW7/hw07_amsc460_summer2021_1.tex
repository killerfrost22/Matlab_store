\documentclass[12pt]{article}
\usepackage{hyperref,color,palatino,amsmath,amssymb,graphicx}
\usepackage[normalem]{ulem} %For striking out text
\usepackage[margin=1in]{geometry}

\hypersetup{
    colorlinks=true,
    linkcolor=blue,
    filecolor=magenta,      
    urlcolor=cyan,
}
\urlstyle{same}

\pagenumbering{gobble}

\begin{document}

\begin{center}
Computational Methods\qquad Summer 2021
\\

\textbf{\large HOMEWORK 7}\\
\end{center}
\noindent \textbf{Due Date}: Friday, June 11\\

\begin{enumerate}
\item Suppose you have used a finite element method to convert a system of PDE to a $10^6\times 10^6$ linear system of equations $A\textbf{x}=\textbf{b}$. To check runtime, you decide to first solve a $100\times 100$ linear system $C\textbf{x}=\textbf{d}$ and find that it takes a total of $0.001$ seconds using an LU decomposition method (elimination $+$ backsolve). 

Give an estimate of the time it will take to solve the system $Ax=b$. Would you consider using LU to solve $A\textbf{x}=\textbf{b}$? Or is a faster method needed? 
[Assume $A$ and $C$ have similar structure, so the difference in runtime is solely due to the different matrix sizes.]
\item Find the $PA=LU$ decomposition (using partial pivoting) for the matrix
	\[\displaystyle{A = \begin{bmatrix}
    2 & 1 \\
    4 & 3 \\
	\end{bmatrix}}.\]
All calculations should be recorded and done by hand. Check your answer using MATLAB's \texttt{lu} command.
\item (Optional, not graded) Find the LU decomposition of 
	\[\displaystyle{A = \begin{bmatrix}
    4 & 2 & 0\\
    4 & 4 & 2 \\
    2 & 2 & 3
	\end{bmatrix}}.\]
\item (Optional, not graded) Suppose $L$ is a nonsingular lower triangular matrix, $P$ is a permutation matrix, and $\textbf{b}$ is a given vector. How would you efficiently solve the following two linear systems? Without using inverses of course... Comment on the operation counts involved. [Consider that permutation matrices $P$ are orthogonal, so we know $P^{-1}=P^T$.]
	\begin{enumerate}
	\item $LP\textbf{x}=\textbf{b}$
	\item $PL\textbf{x}=\textbf{b}$
	\end{enumerate}
\end{enumerate}
\end{document}