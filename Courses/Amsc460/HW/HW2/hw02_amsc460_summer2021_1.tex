\documentclass[12pt]{article}
\usepackage{hyperref,color,palatino,amsmath,amssymb,graphicx}
\usepackage[normalem]{ulem} %For striking out text
\usepackage[margin=1in]{geometry}

\hypersetup{
    colorlinks=true,
    linkcolor=blue,
    filecolor=magenta,      
    urlcolor=cyan,
}
\urlstyle{same}

\pagenumbering{gobble}

\begin{document}

\begin{center}
Computational Methods\qquad Summer 2021 I
\\

\textbf{\large HOMEWORK 2}\\
\end{center}
\noindent \textbf{Due Date}: Friday, June 4\\

\noindent Homework should be handed in \emph{individually}, though you may work with others and collaboration is encouraged. For MATLAB problems please follow the guidelines specified in ELMS (in particular see the file ``Formatting MATLAB assignments'')

\begin{enumerate}\item
\begin{enumerate}
	\item Write a MATLAB program to implement Newton's method for root finding.
	\item To compare root finding algorithms, we will approximate $\sqrt{2}$ using two methods: Newton and Bisection. Using the equation $f(x)=x^2-2=0$, use your program from part (a) to ensure $\sqrt{2}$ is obtained. For Newton, use $x_0=2$, and for Bisection use the starting bracket $[1,2]$. In each case use $10^{-8}$ for the error tolerance. You can use the bisection method code from class.
	\item Modify the algorithms to keep track of the absolute error $e_n = |r - x_n|$ at each iteration. Store these errors in a vector (for plotting purposes). Then plot the absolute errors on the same graph, and with a semilogarithmic y-axis (use \texttt{semilogy} in MATLAB). Which algorithm used the least steps to achieve the required error tolerance?
	\end{enumerate}
\end{enumerate}
\end{document}