\documentclass[12pt]{article}
\usepackage{hyperref,color,palatino,amsmath,amssymb,graphicx}
\usepackage[normalem]{ulem} %For striking out text
\usepackage[margin=1in]{geometry}

\hypersetup{
    colorlinks=true,
    linkcolor=blue,
    filecolor=magenta,      
    urlcolor=cyan,
}
\urlstyle{same}

\pagenumbering{gobble}

\begin{document}

\begin{center}
Computational Methods\qquad Summer 2021
\\

\textbf{\large HOMEWORK 4}\\
\end{center}
\noindent \textbf{Due Date}: Tuesday, June 8\\

\begin{enumerate}
\item Express $x = (12.8)_{10}$ as a normalized double-precision IEEE floating point number $fl(x)$ (i.e. in the form $\pm 1.\textrm{mantissa}\times 2^E$), using the round-to-nearest rule.
\item \begin{enumerate}
\item Explain why between $2^{53}$ and $2^{54}$, the only double precision floating point numbers that exist are the even numbers. 
\item Suppose we type the following into the MATLAB command prompt
	\[x = 2^{53}+1\]
	What will MATLAB store in $x$? Explain. 
\end{enumerate}
\item (Optional, not graded) Express $(12.8)_{10}$ as a \emph{computer word}. (You will need to read the supplementary notes about how exponents are stored using the \emph{exponent bias}.)
\end{enumerate}
\end{document}