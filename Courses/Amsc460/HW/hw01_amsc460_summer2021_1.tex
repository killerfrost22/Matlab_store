\documentclass[12pt]{article}
\usepackage{hyperref,color,palatino,amsmath,amssymb,graphicx}
\usepackage[normalem]{ulem} %For striking out text
\usepackage[margin=1in]{geometry}

\hypersetup{
    colorlinks=true,
    linkcolor=blue,
    filecolor=magenta,      
    urlcolor=cyan,
}
\urlstyle{same}

\pagenumbering{gobble}

\begin{document}

\begin{center}
Computational Methods\qquad Summer 2021 I
\\

\textbf{\large HOMEWORK 1}\\
\end{center}
\noindent \textbf{Due Date}: Thursday, June 3\\

\noindent Homework should be handed in \emph{individually}, though you may work with others and collaboration is encouraged. For MATLAB problems please follow the guidelines specified in ELMS (in particular see the file ``Formatting MATLAB assignments'')

\begin{enumerate}
\item Let $f(x) = e^x + x^2 - 5x$.
	\begin{enumerate}
	\item The bracket $[1.5,2]$ contains a root. Explain why using the Intermediate Value Theorem. For this bracket, estimate the number of iterations $N$ that would be needed to compute the root to an accuracy of $10^{-4}$.
	\item The bracket given in (a) contains a root, but there is another root. Find a bracket for it. Then use the bisection method to find the two roots to an accuracy of $10^{-4}$.
	\end{enumerate}
\item Consider the cubic $f(x)=x^3-x-1$. 
	\begin{enumerate}
	\item Use the MATLAB command \texttt{fzero} to find a root in the interval $[1,2]$.
	\item Show that $f(x)=0$ can be rewritten as a fixed point problem for both the functions (i) $g_1(x)=x^3-1$, and (ii) $g_2(x)=(1+x)^{1/3}$. 
	\item Which of the functions $g_1$ and $g_2$ is a contraction mapping near the root $r$ from part (a)? Which of $g_1$ or $g_2$ will be successful in making the iteration $x_{i+1} = g(x_i)$ converge locally to the root $r$?
	\item Write a script in MATLAB to carry out 10 steps of the fixed point iteration (use a \texttt{for} or \texttt{while} loop to do this) for both $g_1$ and $g_2$, each starting with the guess $x_0=0$. What approximate root does your algorithm give for $g_1$? For $g_2$? Are your results consistent with the analysis from part (c)? 
	\end{enumerate}
\end{enumerate}
\end{document}