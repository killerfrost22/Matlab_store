\documentclass[12pt]{article}
\usepackage{hyperref,color,palatino,amsmath,amssymb,graphicx}
\usepackage[normalem]{ulem} %For striking out text
\usepackage[margin=1in]{geometry}

\hypersetup{
    colorlinks=true,
    linkcolor=blue,
    filecolor=magenta,      
    urlcolor=cyan,
}
\urlstyle{same}

\pagenumbering{gobble}

\begin{document}

\begin{center}
Computational Methods\qquad Summer 2021
\\

\textbf{\large HOMEWORK 5}\\
\end{center}
\noindent \textbf{Due Date}: Wednesday, June 9\\

\begin{enumerate}
\item In Homework 4, problem 1, we computed $x = (12.8)_{10}$ as a double-precision IEEE float $fl(x)$, using the round-to-nearest rule. Now compute the relative error $d=|x-fl(x)|/|x|$ exactly as a base-10 number, and show that $d$ satisfies the upper bound $d\le \displaystyle{\frac{\varepsilon_{mach}}{2}}$. [Do the computations in base 10. In particular treat $fl(x)$ as a normal base 10 number.]
\item Let $x = 2$. To avoid subtraction of nearly equal numbers, find an alternative form $\widetilde{f}(h) \equiv f(h)$ to evaluate 
\begin{equation}
f(h) = \frac{x^4-(x-h)^4}{h}
\end{equation}
for small $h$. Compute $f(h)$ using MATLAB based on the formula (1) and the alternative form $\widetilde{f}(h)$ you propose, and report your results for $h=10^{-1},10^{-2},...,10^{-18}$ on a \texttt{semilogx} plot (both functions should be on the same graph). What is $\lim_{h\rightarrow 0}f(h)$? Does your modified function compute more accurately for small $h$? 

Note: A vector of powers of 10 can be quickly created using \texttt{x = 10*ones(1,15)} to create a vector of 10's, and \texttt{y = 1:1:18} to create the vector of exponents, and finally \texttt{h = x.\^\,(-y)} to get a vector of the required powers of 10.

\item (Optional, not graded) Consider a right triangle whose legs are of length 3344556600 and 1.2222222 (seven 2's). Using MATLAB to compute, how much longer is the hypotenuse than the longer leg? Do this without using the Symbolic Toolbox or VPA!
\end{enumerate}
\end{document}