\documentclass[12pt]{article}
\usepackage{hyperref,color,palatino,amsmath,amssymb,graphicx,enumerate}
\usepackage[normalem]{ulem} %For striking out text
\usepackage[margin=1in]{geometry}

\hypersetup{
    colorlinks=true,
    linkcolor=blue,
    filecolor=magenta,      
    urlcolor=cyan,
}
\urlstyle{same}

\pagenumbering{gobble}

\begin{document}

\begin{center}
Computational Methods\qquad Summer 2021
\\

\textbf{\large HOMEWORK 11}\\
\end{center}
\noindent \textbf{Due Date}: Monday, June 21\\

\begin{enumerate}
\item Write down the polynomial that interpolates $f(x)=e^x$ at the points $x_0=0$, $x_1 = 1$, and $x_2=2$ in Lagrange form.
\item The Vandermonde matrix can be badly conditioned and is not ideal for solving many interpolation problems. On the other hand, some of this ill-conditioning can be mitigated by scaling the data. Suppose we are given data points $(x_0,y_0),..,(x_n,y_n)$ with $x_0<x_1<...<x_n$. Consider scaling the $x$ values by letting 
	\[z_i = \frac{x_i-\alpha}{\beta},\]
where $\alpha$ and $\beta$ are given numbers with $\beta>0$. The data points $(x_i,y_i)$ change to $(z_i,y_i)$, and the interpolation polynomial changes to 
	\[p_n(z) = a_0+a_1 z+\cdot\cdot\cdot+a_n z^n.\]
	\begin{enumerate}
	\item The original data interval is $x_0\le x\le x_n$. What is the data interval when using $z=(x-\alpha)/\beta$? What matrix equation must be solved to find the $a_i's$ in the above formula for $p_n(z)$?
	\item Taking a hint from the previous step, the data will be scaled so that the new data interval is instead $-1\le z\le 1$. What must $\alpha$ and $\beta$ be here?
	\item Consider the following population data for the USA over the 100 year period between 1900 and 2000.
	\begin{center}
  \begin{tabular}{ l | c | c | c | c | c | c | c | c | c | c | c }
    x & 1900 & 1910 & 1920 & 1930 & 1940 & 1950 & 1960 & 1970 & 1980 & 1990 & 2000\\ \hline
    y & 76.21 & 92.23 & 106 & 123.2 & 151.3 & 179.3 & 203.3 & 226.5 & 248.8 & 281.4 & 308.7
  \end{tabular}
\end{center}
	The $y$ values represent the population of the USA in \emph{millions}. Using the direct approach (Vandermonde), plot the interpolation function using the original $x_i$ data. You should use MATLAB's \texttt{vander} command to construct the Vandermonde matrix $V$. Using MATLABs \texttt{cond} commmand, what is the condition number $\mathrm{cond}(V)$ of the associated Vandermonde matrix $V$?
	\item Using the same population data from part (c), scale the data to $[-1,1]$ and find the coefficients for $p_n(z)$. What is the condition number in this case? Once the $a_i's$ are computed the resulting (unscaled) polynomial is 
	\[p_n(x)=a_0+a_1\left(\frac{x-\alpha}{\beta}\right)+\cdot\cdot\cdot a_n\left(\frac{x-\alpha}{\beta}\right)^n.\]
Plot this function and compare it with the function you found in part (c). Comment on the difference between the two.
	\end{enumerate}
\end{enumerate}
\end{document}