\documentclass[12pt]{article}
\usepackage{hyperref,color,palatino,amsmath,amssymb,graphicx,enumerate}
\usepackage[normalem]{ulem} %For striking out text
\usepackage[margin=1in]{geometry}

\hypersetup{
    colorlinks=true,
    linkcolor=blue,
    filecolor=magenta,      
    urlcolor=cyan,
}
\urlstyle{same}

\pagenumbering{gobble}

\begin{document}

\begin{center}
Computational Methods\qquad Summer 2021
\\

\textbf{\large HOMEWORK 13}\\
\end{center}
\noindent \textbf{Due Date}: Wednesday, June 23\\

\begin{enumerate}
\item Suppose in designing a natural logarithm function for a calculator on the interval $[1,e]$, we are using a Chebyshev polynomial approximation. What is the smallest degree $n$ of the polynomial that ensures an accuracy of $10^{-6}$ over the interval $[1,e]$?
\item Special functions appear in physics and applied mathematics, often as a solution to some ODE. The following function is in the \emph{Bessel} family (\url{https://en.wikipedia.org/wiki/Bessel_function})
	\[J(x) = \frac{1}{\pi}\int_0^\pi\!\cos(x\sin(s))\,\mathrm{d}s.\]
	\begin{enumerate}
	\item Show that $|J(x)|\le 1$, $|J'(x)|\le 1$, $|J''(x)|\le 1$, and in general that $|J^{(k)}(x)|\le 1$ for any positive integer $k$.
	\item Suppose we would like to approximate $J$ with a Chebyshev interpolant. Determine how many interpolation points  are required on the interval $[0,10]$ so that the error (in the max-norm) is no more than $10^{-6}$. [You don't have to write down the interpolant.]
	\end{enumerate}
\end{enumerate}
\end{document}