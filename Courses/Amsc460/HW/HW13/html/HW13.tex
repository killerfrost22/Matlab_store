
% This LaTeX was auto-generated from MATLAB code.
% To make changes, update the MATLAB code and republish this document.

\documentclass{article}
\usepackage{graphicx}
\usepackage{color}

\sloppy
\definecolor{lightgray}{gray}{0.5}
\setlength{\parindent}{0pt}

\begin{document}

    
    
\subsection*{Contents}

\begin{itemize}
\setlength{\itemsep}{-1ex}
   \item AMSC 460 - HW11
   \item Problem 1
   \item Problem 2 (
\end{itemize}


\subsection*{AMSC 460 - HW11}

\begin{verbatim}
clear all; format compact; close all; syms f(x) x y z
\end{verbatim}


\subsection*{Problem 1}

\begin{par}
Suppose in designing a natural logarithm function for a calculator on the interval $[1,e]$, we are using a Chebyshev polynomial approximation. What is the smallest degree $n$ of the polynomial that ensures an accuracy of $10^{-6}$ over the interval $[1,e]$?
\end{par} \vspace{1em}
\begin{par}
From Chebyshev polynomial we have [a,b]=[1,e] and f(x)=lnx. The maximum will be(n-1)! where
\end{par} \vspace{1em}
\begin{par}
because the maximum of 1/z on [1, e] occurs at z = 1. We now look for an integer d such that the error is strictly smaller than 10\^{}−6
\end{par} \vspace{1em}
\begin{par}
\[(\frac{1}{n*2^(n-1)} * (\frac{e-1}{2})^n) < 10^(-a)\]
\end{par} \vspace{1em}
\begin{par}
n = 10 error ≤ 1.7*10\^{}−5
\end{par} \vspace{1em}
\begin{par}
n = 13 error ≤ 1.04*10\^{}−6
\end{par} \vspace{1em}
\begin{par}
n = 14 error ≤ 4.17·10\^{}−7
\end{par} \vspace{1em}
\begin{par}
Thus n = 14 is required, the (n-1)th degree polynomial so 13.
\end{par} \vspace{1em}
\begin{par}
Therefpre we need at least n = 15
\end{par} \vspace{1em}

\subsection*{Problem 2}

\begin{par}
Special functions appear in physics and applied mathematics, often as a solution to some ODE. The following function is in the \ensuremath{\backslash}emph\{Bessel\} family (\ensuremath{\backslash}url\{https://en.wikipedia.org/wiki/Bessel\_function\})[J(x) = \ensuremath{\backslash}frac\{1\}\{\ensuremath{\backslash}pi\}\ensuremath{\backslash}int\_0\^{}\ensuremath{\backslash}pi\ensuremath{\backslash}!\ensuremath{\backslash}cos(x\ensuremath{\backslash}sin(s))\ensuremath{\backslash},\ensuremath{\backslash}mathrm\{d\}s.\ensuremath{\backslash}]
\end{par} \vspace{1em}
\begin{par}
(a)Show that $|J(x)|\le 1$, $|J'(x)|\le 1$, $|J''(x)|\le 1$, and in general that $|J^{(k)}(x)|\le 1$ for any positive integer $k$.
\end{par} \vspace{1em}
\begin{par}
n is from the even number above 0 that cos and sin both fluctuate bewteen 0 and 1 which meaning both sin and cosine will be bounded between 0 and 1, similarly for odd numbers that will also be bounded by 1 and 0
\end{par} \vspace{1em}

\begin{par}
$|J(x)|\le 1$
\end{par} \vspace{1em}
\begin{par}
(b)Suppose we would like to approximate $J$ with a Chebyshev interpolant. Determine how many interpolation points  are required on the interval $[0,10]$ so that the error (in the max-norm) is no more than $10^{-6}$. [You don't have to write down the interpolant.]
\end{par} \vspace{1em}

\begin{par}
Since $|J(x)|\le 1$ Given values smaller than 10^(-6). for \[\frac{5^n}{(n!)(2^(n-1))}\]
\end{par} \vspace{1em}

\begin{verbatim}
   for (n = 0; (5^n)/(factorial(n)(2^(n-1))) > 10^(-6); n++)
   \end{verbatim}
           \color{lightgray} \begin{verbatim}f(x) =
   (exp(3*x)*sin(200*x^2))/(20*x^2 + 1)
\end{verbatim} \color{black}   

\color{lightgray} \begin{verbatim}z =
   n =16
\end{verbatim} \color{black}
\end{document}

