\documentclass{article}
\usepackage[utf8]{inputenc}
\usepackage[margin=1in]{geometry}

\usepackage{epsfig,amssymb,amsmath,multicol,xcolor,enumerate}

\newcommand{\N}{\mathbb{N}}
\newcommand{\Z}{\mathbb{Z}}
\newcommand{\Q}{\mathbb{Q}}
\newcommand{\R}{\mathbb{R}}
\newcommand{\C}{\mathbb{C}}
\title{HW2}
\author{Yizhan Ao}
\date{11 September 2021}

\begin{document}

\maketitle 

\section{Question 1}
\item \textbf{(Must do all computations by hand.)} Consider the linear transformation $f: \R^2 \to \R^2$ which first reflects through the $y$-axis, and then rotates counterclockwise about the origin by $\pi/4$ radians (45 degrees).  Find the standard $2 \times 2$ matrix for this linear transformation in two ways:
	\begin{enumerate}
		\item Find the standard matrix of $f$ by directly determining the images of the standard basis vectors $\mathbf{e}_1$ and $\mathbf{e}_2$ under $f$.
		\item Find the standard matrix of $f$ by first finding the standard matrices of the reflection and of the rotation, and then use matrix multiplication.
	\end{enumerate}
\item \textbf{1 a) }
\begin{center}
\item Let $f_1$ be a reflection function that through the y-axis, let $f_2$ be a rotation counter clockwise $\pi/4$ about the origin we can get 
\item $f_1(x,y) = (x,-y)$
\item $f_2(x,y) = \begin{bmatrix} x*cos(\pi/4)-y*sin(\pi/4)\\ x*sin(\pi/4)+y*cos(\pi/4) \end{bmatrix} $
\item when we are consider the reflection $\mathbf{e}_1$ = \begin{bmatrix} 1\\0\end{bmatrix} through y-axis it maps (1,0) then rotaton $\pi/4$ and we can get $(cos(\pi/4) , sin(\pi/4)) = (\frac{1}{\sqrt2}, \frac{1}{\sqrt2})$
\item so  $\mathbf{e}_1$ maps to $(\frac{1}{\sqrt2}, \frac{1}{\sqrt2})$ under $F_1$
\item Similarly  $\mathbf{e}_2$ maps to $(0,-1)$ so F2 maps to $(-1*x*sin(\pi/4)-1*y*cos(\pi/4))$ 
\item Therefore, $F(\mathbf{e}_2) = (-\frac{1}{\sqrt2}, -\frac{1}{\sqrt2})$
\item $ F = \begin{bmatrix}-\frac{1}{\sqrt2} & -\frac{1}{\sqrt2}\\ -\frac{1}{\sqrt2} &\frac{1}{\sqrt2}\end{bmatrix}$
\end{center}

\item \textbf{1 b) }
\begin{center}
\item $F_1 = \begin{bmatrix}\ -1 & 0 \\ 0 & 1\end{bmatrix}$
\item $F_2 = \begin{bmatrix}\frac{1}{\sqrt2} & -\frac{1}{\sqrt2}\\ \frac{1}{\sqrt2} &\frac{1}{\sqrt2}\end{bmatrix}$
\item $F_1 * F_2 = F_1 = \begin{bmatrix}\ -1 & 0 \\ 0 & 1\end{bmatrix} * \begin{bmatrix}\frac{1}{\sqrt2} & -\frac{1}{\sqrt2}\\ \frac{1}{\sqrt2} &\frac{1}{\sqrt2}\end{bmatrix} = \begin{bmatrix}-\frac{1}{\sqrt2} & -\frac{1}{\sqrt2}\\ -\frac{1}{\sqrt2} & \frac{1}{\sqrt2}\end{bmatrix}$
\item 
\end{center}

\section{Question 2}
\item Find the point obtained when $(7, 9)$ is rotated counterclockwise about the origin by 50 degrees ($5\pi/18$ radians).  (Do the computation in MATLAB, but make it clear what computation you are doing.)
\begin{equation}
\left[\begin{array}{ccc}
\cos \left(\frac{5 \pi}{18}\right) & -\sin \left(\frac{5 \pi}{18}\right) & 0 \\
\sin \left(\frac{5 \pi}{18}\right) & \cos \left(\frac{5 \pi}{18}\right) & 0 \\
0 & 0 & 1
\end{array}\right]\left[\begin{array}{c}
7 \\
9 \\
1
\end{array}\right]=\left[\begin{array}{c}
-2.395\\11.147 \\ 1
\end{array}\right]
\end{equation}
\begin{center}
    

\item the point is (-2.395, 11.147)
\end{center}

\begin{verbatim}
Code proof
format long;
x = 7, y =9;
r = sqrt(x^2+y^2)
Q = atan(y/x)
X = 7*sind(50)-9*sind(50)
Y = 9*cosd(50)+7*sind(50)
\item r =  11.401754250991379
\item Q =  0.909753157944210
\item X =  -2.395088886237956
\item Y =  11.147399589011700
\end{verbatim}

\section{Question 3}
\item In two dimensions write down the matrix which rotates around
the origin by $\pi/6$ radians and then translates by -3 in the x-direction and 5 in
the y-direction. Apply this matrix to the points (0; 3) and (1,1).

\begin{equation}
T(-3,5) R\left(\frac{\pi}{6}\right)=\left[\begin{array}{ccc}
1 & 0 & -3 \\
0 & 1 & 5 \\
0 & 0 & 1
\end{array}\right]\left[\begin{array}{ccc}
\cos \left(\frac{\pi}{b}\right) & -\sin \left(\frac{\pi}{6}\right) & 0 \\
\sin \left(\frac{\pi}{6}\right) & \cos \left(\frac{\pi}{6}\right) & 0 \\
0 & 0 & 1
\end{array}\right]
\end{equation}


\item \textbf{3 a) }
\begin{equation}
T(-3,5) R\left(\frac{\pi}{6}\right)\begin{bmatrix}
0\\3\\1
\end{bmatrix} = (-\frac{9}{2}, \frac{10+3\sqrt{2}}{2})
\end{equation}
\item Therefore, (x',y') after rotaion is $(-\frac{9}{2}, \frac{10+3\sqrt{2}}{2})$
\item \textbf{3 b) }
\begin{equation}
T(-3,5) R\left(\frac{\pi}{6}\right)\begin{bmatrix}
1\\-1\\1
\end{bmatrix} = (-\frac{\sqrt{3}-5}{2}, \frac{9-\sqrt{3}}{2})
\end{equation}
\item Therefore, (1,-1) after rotaion is $(-\frac{\sqrt{3}-5}{2}, \frac{9-\sqrt{3}}{2})$

\section{Question 4}
\begin{enumerate}
\item In two dimensions find the image of the three points (6; 3); (4; 1); (7; 1)
under rotation around the point (8; 2) by $\pi=3$ radians. Sketch the original points
and the images.

 $$T(8, 2)R(\pi/3)T(-8, −2) =\left[ \begin{matrix} 1 & 0 & 8 \\ 0 &1 & 2\\ 0 & 0 &1 \end{matrix} \right] \left[ \begin{matrix} cos(\pi/3) & -sin(\pi/3) &0 \\ sin(\pi/3)&cos(\pi /3) & 0\\ 0&0 &1 \end{matrix} \right]\left[ \begin{matrix} 1 & 0 &-8 \\ 0&1 & -2\\ 0&0 &1 \end{matrix} \right]$$$$= \left[ \begin{matrix} 0.5000 & -0.8660 &5.7321 \\ 0.8660&0.5000 & -5.9282\\ 0&0 &1 \end{matrix} \right]$$ $$\left[ \begin{matrix} 0.5000 & -0.8660 &5.7321 \\ 0.8660&0.5000 & -5.9282\\ 0&0 &1 \end{matrix} \right]\left[ \begin{matrix} 6& 4 &7\\ 3&1& 1\\ 1&1 &1 \end{matrix} \right]=\left[ \begin{matrix} 6.1340 & 6.8660 & 8.3660 \\ 0.7679 & -1.9641 & 0.6340\\ 1&1 &1 \end{matrix} \right]$$
\includegraphics{sc2.9.png}
\end{enumerate}
\section{Question 5}
\begin{enumerate}
    \item In two dimensions write down the matrix (simplified) for rotation around the point (a; b) by $\theta$ radians.
\end{enumerate}
\begin{equation}
T(a, b) R(\theta) T(-a,-b)=\left[\begin{array}{lll}
1 & 0 & a \\
0 & 1 & b \\
0 & 0 & 1
\end{array}\right]\left[\begin{array}{ccc}
\cos \theta & -\sin \theta & 0 \\
\sin \theta & \cos \theta & 0 \\
0 & 0 & 1
\end{array}\right]\left[\begin{array}{ccc}
1 & 0 & -a \\
0 & 1 & -b \\
0 & 0 & 1\end{array}\right]
\end{equation}

\section{Question 6}
Find the 3  3 matrix (for homogeneous coordinates) for the transformation which rst rotates counterclockwise about the point (4; 8) by =5 radians, and then rotates counterclockwise about the point (7; 1) by 4=11 radians.

\begin{center}

\item $T_1$ = 
\begin{bmatrix}
cos(\pi/5) & -sin(\pi/5)  & 4\\ 
sin(\pi/5) & cos(\pi/5) & 8\\ 
0 & 0 & 1
\end{bmatrix}

\item $T_2$ = 
\begin{bmatrix}
cos(\pi/5) & -sin(\pi/5)  & 4\\ 
sin(\pi/5) & cos(\pi/5) & 8\\ 
0 & 0 & 1
\end{bmatrix}

\item
$T_1$ * $T_2$ = 
\begin{bmatrix}
-0.507 & -1.122 & 1.394\\ 
1.122 & -0.507 & 7.969\\ 
0 & 0 & 1
\end{bmatrix}
\end{center}

\section{Question 7}
\begin{enumerate}
	\item Let $L$ denote a line in the plane that does \textbf{not} go through the origin.  Let $f: \R^2 \to \R^2$ denote the transformation which reflects through the line $L$.  Give a reason why $f$ is not a linear transformation.
	\item Reflection through a line which does not go through the origin can be represented as a $3 \times 3$ matrix (which operates on homogeneous coordinates).  Give the matrix for reflection through the vertical line $x=5$.
	\item Give a formula for the image of the generic point $(x,y)$ under reflection through the line $x=5$.
\end{enumerate}

\item \textbf{1 ) }
\begin{center}  
\item let L be a line within the coordinates that doesnt pass through the origin, let ax + by = c and c not equal to 0
\item so we can be the perpendicular line to L that f(0,0) = 2*A that doesnt equal to (0,0) 
\item therefore f is not a line  of transformation
\end{center}
\item \textbf{2 ) }
\begin{center}
\item let line L be x = 5and y coordinate will remain the same but at the x coordinate 
\item x ->10-x 
\item So f(x,y) = (10-x, y) which can also be (10, 0) + (-x, y)
\item so the matrix will be A = 
\begin{bmatrix}
10 & 0 & 0
\end{bmatrix}
\end{center}
+ \begin{bmatrix}
-1 & 0 & 0\\ 
0 & 1 & 0\\ 
0 & 0 & 0
\end{bmatrix}$
\item $f(x,y) = \begin{bmatrix}
10 & 0 & 0
x}
\end{bmatrix} + 
\begin{bmatrix}
x & y & 0
\end{bmatrix}
\item \textbf{3 ) }
\begin{center}
    c) from the above we can have $f(x,y) = (10-x,y)$
\end{center}

\section{Question 8}
\item \textbf{(Must do all computations by hand.)} Let $L_\theta$ denote the line in the plane through the origin that makes an angle of $\theta$ radians with the positive $x$-axis.  In class, we saw that reflection through the line $L_\theta$ has standard matrix $\begin{bmatrix}
\cos(2\theta) & \sin(2\theta)\\
\sin(2\theta) & -\cos(2\theta)
\end{bmatrix}$.  Now let $\theta$ and $\varphi$ be two angles.  Show that the composition of reflection through $L_\varphi$ followed by reflection through $L_\theta$ is equal to a rotation about the origin, and find the angle of rotation.  (Multiply the matrices and use trig identities to recognize the result as a rotation matrix.)

\begin{equation}
\begin{aligned}
&\left[\begin{array}{lll}
\cos \theta & \sin \theta \\
\sin \theta & \operatorname{ars} \theta
\end{array}\right]\left[\begin{array}{cc}
\cos \theta & \sin \alpha \\
\sin \alpha & -\cos \alpha
\end{array}\right]= \left[\begin{array}{lll}
\cos \theta \cos \theta+\sin \theta \cos \alpha & \cos \theta \sin \alpha-\sin \theta \cos \alpha \\
\sin \theta \cos 2 d-\cos \theta \operatorname{cin} \alpha & \sin \theta \sin \alpha+\cos \theta \cos \alpha
\end{array}\right] 
\end{aligned}
\end{equation}

\item  = $\begin{equation}
\left[\begin{array}{ccc}
\cos (\theta-\alpha) & -\sin (\theta-\alpha) \\
\sin (\theta-a) & \cos (\theta-\alpha)
\end{array}\right]
\end{equation}$

\begin{center}
  \item Therefore we can comeup to a conclusion that the angle is $\theta - \alpha}$  
\end{center}


\end{document}
